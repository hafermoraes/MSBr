
\documentclass[../../relatorio.tex]{subfiles}

\begin{document}


\section{Seguro de Danos}

Segundo definição do órgão regulador do mercado, a SUSEP, o objetivo do seguro de danos é garantir ao segurado, até o limite máximo de garantia e de acordo com as condições do contrato, o pagamento de indenização por prejuízos, devidamente comprovados, diretamente decorrentes de perdas e/ou danos causados aos bens segurados, ocorridos no local segurado, em conseqüência de risco coberto. Estes são os diferentes tipos de Seguros de Danos no mercado brasileiro:

\begin{description}
  \item[Aeronáuticos] Seguros de responsabilidade civil facultativa para aeronaves, aeronáuticos, responsabilidade civil de hangar e responsabilidade do explorador ou transportador aéreo
  \item[Automóvel] Seguros contra roubos e acidentes de carros, de responsabilidade civil contra terceiros e DPVAT
  \item[Cascos (em "run-off")] Seguros contra riscos marítimos, aeronáuticos e de hangar
  \item[Crédito (em "run-off")] Seguros de crédito à exportação e contra riscos comerciais e políticos
  \item[Habitacional] Seguros contra riscos de morte e invalidez do devedor e de danos físicos ao imóvel financiado
  \item[Marítimos] Seguros compreensivos para operadores portuários, responsabilidade civil facultativa para embarcações e marítimos
  \item[Microsseguros] Microsseguros de pessoas, danos e previdência
  \item[Outros] Seguros no exterior e de sucursais de seguradores no exterior
  \item[Patrimonial] Seguros contra incêndio, roubo de imóveis bem como os seguros compreensivos residenciais, condominiais e empresariais
  \item[Responsabilidades] Seguros contra indenizações por danos materiais ou lesões corporais a terceiros por culpa involuntária do segurado
  \item[Riscos Especiais] Seguros contra riscos de petróleo, nucleares e satélites
  \item[Riscos Financeiros] Seguros diversos de garantia de contratos e de fiança locatícia
  \item[Rural] Seguros agrícola, pecuário, de florestas e penhor rural
  \item[Transportes] Seguros de transporte nacional e internacional e de responsabilidade civil de cargas, do transportador e do operador
\end{description}

Até Maio de 2015, o mercado de Seguro de Danos totaliza BRL 27 Bilhões (contra BRL 26.2 Bilhões no mesmo período do ano passado). Em dezembro do ano anterior o mercado de Seguro de Danos representava aproximadamente 40.7\% do mercado segurador brasileiro (com BRL 62.8 Bilhões de prêmios).

\pagebreak

% latex table generated in R 3.1.1 by xtable 1.7-4 package
% Tue Jul 28 19:04:00 2015
\begin{table}[H]
\centering
\begin{tabular}{lrrrrrrrrrr}
  \hline
 & 2006 & 2007 & 2008 & 2009 & 2010 & 2011 & 2012 & 2013 & 2014 & 2015 \\ 
  \hline
Aeronáuticos &  &  &  &  &  & 0,27 & 0,35 & 0,38 & 0,40 & 0,19 \\ 
  Automóvel & 15,79 & 17,13 & 19,82 & 19,78 & 22,60 & 24,36 & 27,94 & 33,12 & 39,38 & 17,55 \\ 
  Cascos & 0,33 & 0,40 & 0,50 & 0,56 & 0,56 & 0,05 & 0,01 &  &  &  \\ 
  Crédito & 0,56 & 0,54 & 0,50 & 0,43 & 0,43 & 0,21 & 0,15 & 0,11 & 0,09 & 0,02 \\ 
  Habitacional & 0,50 & 0,56 & 0,72 & 0,91 & 1,11 & 0,98 & 0,99 & 1,01 & 1,03 & 0,45 \\ 
  Marítimos &  &  &  &  &  & 0,26 & 0,26 & 0,35 & 0,36 & 0,15 \\ 
  Microsseguros &  &  &  &  &  &  &  &  &  &  \\ 
  Outros &  &  &  &  &  &  &  & - &  &  \\ 
  Patrimonial & 4,91 & 5,53 & 6,37 & 6,39 & 7,80 & 9,28 & 9,91 & 11,46 & 12,27 & 4,98 \\ 
  Responsabilidades & 0,46 & 0,53 & 0,60 & 0,64 & 0,75 & 0,93 & 1,07 & 1,24 & 1,29 & 0,63 \\ 
  Riscos Especiais & 0,19 & 0,28 & 0,20 & 0,24 & 0,17 & 0,41 & 0,51 & 0,72 & 0,59 & 0,27 \\ 
  Riscos Financeiros & 0,25 & 0,44 & 0,66 & 0,87 & 0,92 & 1,28 & 1,45 & 1,90 & 2,14 & 0,99 \\ 
  Rural & 0,33 & 0,45 & 0,71 & 0,90 & 0,87 & 1,04 & 1,25 & 1,98 & 2,46 & 0,61 \\ 
  Transportes & 1,46 & 1,58 & 1,85 & 1,69 & 1,97 & 2,41 & 2,61 & 2,88 & 2,74 & 1,17 \\ 
   \hline
Total & 24,79 & 27,46 & 31,94 & 32,40 & 37,18 & 41,49 & 46,50 & 55,16 & 62,75 & 27,01 \\ 
   \hline
\% do Mercado & 34,09 & 33,16 & 33,20 & 30,09 & 29,66 & 28,39 & 26,47 & 28,18 & 40,66 & 51,78 \\ 
   \hline
\% do PIB & 1,03 & 1,01 & 1,03 & 0,97 & 0,96 & 0,95 & 0,99 & 1,07 & 1,14 & 1,92 \\ 
   \hline
\end{tabular}
\caption{Evolução do Mercado de Seguro de Danos. Valores em BRL Bilhões} 
\end{table}
O mercado de Seguro de Danos é dominado cada vez menos pelas seguradoras ligadas aos bancos: 34.5\% dos prêmios até Maio de 2015 ; há 5 anos elas detinham 47.5\% e há 10 anos 51.3\%.\\

\begin{table}[!h]
  \begin{minipage}[t]{0.49\linewidth}
    \centering
% latex table generated in R 3.1.1 by xtable 1.7-4 package
% Tue Jul 28 19:04:01 2015
\begin{tabular}{llrr}
  \hline
 & Grupo & P & MS \\ 
  \hline
1 & LÍDER & 4,77 & 17,7 \\ 
  2 & MAPFRE-BB & 3,42 & 12,7 \\ 
  3 & PORTO SEGURO & 3,19 & 11,8 \\ 
  4 & BRADESCO & 2,07 & 7,7 \\ 
  5 & ITAÚ & 1,77 & 6,6 \\ 
  6 & SUL AMÉRICA & 1,44 & 5,3 \\ 
  7 & TOKIO MARINE & 1,29 & 4,8 \\ 
  8 & HDI & 1,11 & 4,1 \\ 
  9 & LIBERTY & 1,10 & 4,1 \\ 
  10 & ZURICH & 1,03 & 3,8 \\ 
   \hline
 & Top5 & 15,22 & 56,4 \\ 
   & Top10 & 21,18 & 78,4 \\ 
   & Mercado & 27,01 & 100,0 \\ 
   \hline
\end{tabular}    \captionof{table}{Prêmios em BRL Bilhões (2015/05)}
  \end{minipage}
  \hspace{0.5cm}
  \begin{minipage}[t]{0.49\linewidth}
    \centering
% latex table generated in R 3.1.1 by xtable 1.7-4 package
% Tue Jul 28 19:04:01 2015
\begin{tabular}{llrr}
  \hline
 & Grupo & P & MS \\ 
  \hline
1 & MAPFRE-BB & 8,81 & 14,0 \\ 
  2 & LÍDER & 8,46 & 13,5 \\ 
  3 & PORTO SEGURO & 7,35 & 11,7 \\ 
  4 & ITAÚ & 5,83 & 9,3 \\ 
  5 & BRADESCO & 5,15 & 8,2 \\ 
  6 & SUL AMÉRICA & 3,26 & 5,2 \\ 
  7 & TOKIO MARINE & 2,65 & 4,2 \\ 
  8 & HDI & 2,56 & 4,1 \\ 
  9 & ALLIANZ & 2,53 & 4,0 \\ 
  10 & LIBERTY & 2,45 & 3,9 \\ 
   \hline
 & Top5 & 35,60 & 56,7 \\ 
   & Top10 & 49,05 & 78,2 \\ 
   & Mercado & 62,75 & 100,0 \\ 
   \hline
\end{tabular}    \captionof{table}{Prêmios em BRL Bilhões (2014/12)}
  \end{minipage}
\end{table}

\pagebreak

\end{document}
