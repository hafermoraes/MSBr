\documentclass[../../relatorio.tex]{subfiles}

\begin{document}


\section{Seguro de Saúde}

O Seguro de Saúde tem por objetivo cobrir blabla blá...


% latex table generated in R 3.1.1 by xtable 1.7-4 package
% Tue Jul 28 19:04:01 2015
\begin{table}[ht]
\centering
\begin{tabular}{lrrrrrrrrrr}
  \hline
 & 2006 & 2007 & 2008 & 2009 & 2010 & 2011 & 2012 & 2013 & 2014 & 2015 \\ 
  \hline
\hline
Seguro Saúde & 8,75 & 8,61 & 11,05 & 12,40 & 14,10 & 16,71 & 18,67 & 24,32 & 28,47 &  \\ 
   \hline
\% do Mercado & 12,03 & 10,40 & 11,49 & 11,52 & 11,25 & 11,43 & 10,63 & 12,43 & 18,44 &  \\ 
   \hline
\% do PIB & 0,36 & 0,32 & 0,36 & 0,37 & 0,36 & 0,38 & 0,40 & 0,47 & 0,52 &  \\ 
   \hline
\end{tabular}
\caption{Evolução do Mercado de Seguro de Saúde. Valores em BRL Bilhões} 
\end{table}
Tendo em vista o atraso de divulgação das informações de mercado pela ANS, somente é apresentado o ranking em dezembro do ano anterior. Em dezembro do ano anterior o mercado de Seguro de Saúde representava aproximadamente 18.4\% do mercado segurador brasileiro (com BRL 28.5 Bilhões de prêmios).\\


\begin{table}[!h]
  \begin{minipage}[t]{0.49\linewidth}
    \centering
% latex table generated in R 3.1.1 by xtable 1.7-4 package
% Tue Jul 28 19:04:01 2015
\begin{tabular}{llrr}
  \hline
 & Grupo & P & MS \\ 
  \hline
1 & BRADESCO & 13,66 & 48,0 \\ 
  2 & SUL AMÉRICA & 9,45 & 33,2 \\ 
  3 & UNIMED & 1,90 & 6,7 \\ 
  4 & PORTO SEGURO & 1,05 & 3,7 \\ 
  5 & ALLIANZ & 0,80 & 2,8 \\ 
  6 & YASUDA-MARITIMA & 0,56 & 2,0 \\ 
  7 & NOTRE DAME & 0,53 & 1,8 \\ 
  8 & CAIXA & 0,27 & 1,0 \\ 
  9 & ITAÚ & 0,13 & 0,5 \\ 
  10 & TEMPO & 0,07 & 0,2 \\ 
   \hline
 & Top5 & 26,86 & 94,4 \\ 
   & Top10 & 28,42 & 99,8 \\ 
   & Mercado & 28,47 & 100,0 \\ 
   \hline
\end{tabular}    \captionof{table}{Prêmios em BRL Bilhões (2014/12)}
  \end{minipage}
\end{table}

\pagebreak
\end{document}
