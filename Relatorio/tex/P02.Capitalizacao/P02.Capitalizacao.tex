\documentclass[../../relatorio.tex]{subfiles}

\begin{document}


\section{Capitalização}

O produto Capitalização destina-se a facilitar a poupança ou economia programada do segurado, tornando-o mais disciplinado e oferecendo o benefício do sorteio, regulamentado pela SUSEP, Conselho Nacional de Seguros Privados (CNSP) e pelo Conselho Monetário Nacional (CMN). Ao final do período o segurado recebe o valor total contribuído corrigido pela TR, que pode variar de 20\% a 100\% da taxa da caderneta de poupança.

% latex table generated in R 3.1.1 by xtable 1.7-4 package
% Tue Jul 28 19:04:00 2015
\begin{table}[ht]
\centering
\begin{tabular}{lrrrrrrrrrr}
  \hline
 & 2006 & 2007 & 2008 & 2009 & 2010 & 2011 & 2012 & 2013 & 2014 & 2015 \\ 
  \hline
\hline
Capitalização & 7,11 & 7,83 & 9,02 & 10,10 & 11,78 & 14,08 & 16,59 & 20,98 & 21,88 & 8,28 \\ 
   \hline
\% do Mercado & 9,78 & 9,46 & 9,37 & 9,38 & 9,40 & 9,64 & 9,44 & 10,72 & 14,18 & 15,88 \\ 
   \hline
\% do PIB & 0,30 & 0,29 & 0,29 & 0,30 & 0,30 & 0,32 & 0,35 & 0,41 & 0,40 & 0,59 \\ 
   \hline
\end{tabular}
\caption{Evolução do Mercado de Capitalização. Valores em BRL Bilhões} 
\end{table}
Até Maio de 2015, o mercado de Capitalização totaliza BRL 8.3 Bilhões (contra BRL 8.7 Bilhões no mesmo período do ano passado). Em dezembro do ano anterior o mercado de Capitalização representava aproximadamente 14.2\% do mercado segurador brasileiro (com BRL 21.9 Bilhões de prêmios).

O mercado de capitalização ainda é dominado pelas seguradoras ligadas aos bancos: 82.7\% dos prêmios até Maio de 2015 ; há 5 anos elas detinham 77.7\% e há 10 anos 83.3\%.

\begin{table}[!h]
  \begin{minipage}[t]{0.49\linewidth}
    \centering
% latex table generated in R 3.1.1 by xtable 1.7-4 package
% Tue Jul 28 19:04:00 2015
\begin{tabular}{llrr}
  \hline
 & Grupo & P & MS \\ 
  \hline
1 & BRASILCAP & 2,44 & 29,5 \\ 
  2 & BRADESCO & 2,15 & 26,0 \\ 
  3 & ITAÚ & 1,08 & 13,1 \\ 
  4 & CAIXA & 0,49 & 6,0 \\ 
  5 & SANTANDER & 0,45 & 5,5 \\ 
  6 & SUL AMÉRICA & 0,35 & 4,3 \\ 
  7 & ICATU & 0,33 & 3,9 \\ 
  8 & INVEST & 0,26 & 3,1 \\ 
  9 & HSBC & 0,22 & 2,7 \\ 
  10 & LIDERANCA & 0,16 & 2,0 \\ 
   \hline
 & Top5 & 6,63 & 80,0 \\ 
   & Top10 & 7,95 & 96,0 \\ 
   & Mercado & 8,28 & 100,0 \\ 
   \hline
\end{tabular}    \captionof{table}{Prêmios em BRL Bilhões (2015/05)}
  \end{minipage}
  \hspace{0.5cm}
  \begin{minipage}[t]{0.49\linewidth}
    \centering
% latex table generated in R 3.1.1 by xtable 1.7-4 package
% Tue Jul 28 19:04:00 2015
\begin{tabular}{llrr}
  \hline
 & Grupo & P & MS \\ 
  \hline
1 & BRASILCAP & 6,69 & 30,6 \\ 
  2 & BRADESCO & 5,34 & 24,4 \\ 
  3 & ITAÚ & 2,44 & 11,1 \\ 
  4 & SUL AMÉRICA & 2,07 & 9,4 \\ 
  5 & CAIXA & 1,20 & 5,5 \\ 
  6 & SANTANDER & 1,09 & 5,0 \\ 
  7 & APLUB & 0,91 & 4,2 \\ 
  8 & ICATU & 0,77 & 3,5 \\ 
  9 & HSBC & 0,61 & 2,8 \\ 
  10 & LIDERANCA & 0,36 & 1,6 \\ 
   \hline
 & Top5 & 17,74 & 81,1 \\ 
   & Top10 & 21,48 & 98,2 \\ 
   & Mercado & 21,88 & 100,0 \\ 
   \hline
\end{tabular}    \captionof{table}{Prêmios em BRL Bilhões (2014/12)}
  \end{minipage}
\end{table}

\pagebreak

\end{document}
